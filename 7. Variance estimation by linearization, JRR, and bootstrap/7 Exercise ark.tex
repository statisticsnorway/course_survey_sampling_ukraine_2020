\documentclass[12pt]{article}

% Packages
\usepackage[margin=0.6in]{geometry}
\usepackage{graphicx}
\usepackage{amsmath}
\usepackage{float}
\usepackage[dvipsnames]{xcolor}
\usepackage{etoolbox}
\usepackage{soul}
%\usepackage{tcolorbox}
%\definecolor{block-gray}{gray}{0.85}

\pagestyle{plain}


\begin{document}

\title{Exercise ark \#7.\\ Variance estimation}
\author{O\u{g}uz--Alper, Melike \& Pekarskaya, Tatsiana, Statistics Norway}
\maketitle

\section*{Exercise 1}
\textbf{\color{ForestGreen}(R code available)} Use the data in srs30.dat, which includes $n=30$ units selected with simple random sampling from an artificial population of size $N=100$. 
\begin{enumerate}
\item Use the jackknife to estimate $V(\bar{y})$, and verify that $\hat{V}_{JK}(\bar{y})=s^2/30$ for this data. What are the jackknife weights for jackknife replicate $j$? 
\item Find also the bootstrap estimate of $V(\bar{y})$. \hfill(Lohr, 2019, p.394)
\end{enumerate}

\section*{Exercise 2}
\textbf{\color{ForestGreen}(R code available)} The file agsrs.dat contains data from an SRS of $300$ of the $3\,078$ counties.
Let $y_i$ be total acreage of farms in county $i$ in 1992 and $x_i$ be total acreage of farms in county $i$ in 1987. Use the jackknife and the bootstrap to estimate the variance of the ratio estimator $\hat{B}_r=\bar{y}/\bar{x}$. How do they compare with the linearization estimator? \hfill(Lohr, 2019, pp.394)


\section*{Exercise 3}
\textbf{\color{ForestGreen}(R code available)} Foresters want to estimate the average age of trees in a stand. Determining age is
cumbersome, because one needs to count the tree rings on a core taken from the
tree. In general, though, the older the tree, the larger the diameter, and diameter
is easy to measure. The foresters measure the diameter of all $1\,132$ trees and find
that the population mean equals $10.3$. They then randomly select $20$ trees for age
measurement.
\begin{center}
\begin{tabular}{rrrrrr}
Tree No. & Diameter, $x$ & Age, $y$& Tree No.& Diameter, $x$ &Age, $y$ \\
\hline
1& 12.0& 125 &11 &5.7 &61\\
2& 11.4& 119 &12& 8.0 &80\\
3& 7.9 &83 &13 &10.3& 114\\
4& 9.0 &85& 14 &12.0& 147\\
5& 10.5& 99& 15 &9.2 &122\\
6& 7.9 &117& 16 &8.5& 106\\
7& 7.3& 69 &17 &7.0& 82\\
8& 10.2& 133& 18& 10.7 &88\\
9& 11.7& 154 &19 &9.3 &97\\
10& 11.3& 168 &20& 8.2& 99\\
\end{tabular}
\end{center}
Using the jackknife and the bootstrap, estimate the standard error for the regression estimate of the
population age of trees in a stand. How
do the jackknife and bootstrap compare with the standard error calculated using linearization
methods? \hfill (Lohr, 2019, p.394)

\section*{Exercise 4}
\textbf{\color{ForestGreen}(R code available)} The American Statistical Association (ASA) studied whether it should offer a certification
designation for its members, so that statisticians meeting the qualifications
could be designated as “Certified Statisticians.” In 1994, the ASA surveyed its membership
about this issue, with data in file certify.dat. The survey was sent to all 18 609
members; 5 001 responses were obtained. Results from the survey were reported in
the October 1994 issue of Amstat News.

Assume that in 1994, the ASA membership had the following characteristics: $55\%$
have PhD’s and $38\%$ have Master’s degrees; $29\%$ work in industry, $34\%$ work in
academia, and $11\%$ work in government. The cross-classification between education
and workplace was unavailable. There is a nonresponse. However, treat as if the respondents were selected with a probability sampling from the list of all members. Find the raking estimate of the total number of ASA members for this exercise. Estimate the total number of members opposing certification in 1994 using and use the bootstrap to estimate its variance and construct a $95\%$ confidence interval.\hfill (Lohr, 2019, p.394)

\end{document}