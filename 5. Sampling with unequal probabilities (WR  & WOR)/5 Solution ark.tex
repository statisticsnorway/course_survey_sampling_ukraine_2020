\documentclass[12pt]{article}

% Packages
\usepackage[margin=0.6in]{geometry}
\usepackage{graphicx}
\usepackage{amsmath}
\usepackage{float}
\usepackage[dvipsnames]{xcolor}
\usepackage{etoolbox}
\usepackage{soul}
%\usepackage{tcolorbox}
%\definecolor{block-gray}{gray}{0.85}

\pagestyle{plain}


\begin{document}

\title{Solution ark \#5.\\ Sampling with unequal probabilities}
\author{O\u{g}uz--Alper, Melike \& Pekarskaya, Tatsiana, Statistics Norway}
\maketitle

\section*{Exercise 1}
We have a population of 4 companies. A variable of interest is yearly turnover $y$. Assume that turnover for a given year for the companies 1,2,3,4 is 100, 200, 300 and 1000 millions Norwegian kr. Number employees (x) in each company is known in advance from a register. Assume that x is 20, 30, 50 and 200 for the companies. We are going to samples of size 2 using different methods to estimate the total turnover(as we know the true value is 1600). In (1)-(3) we will find estimators which does not use additional information x. In (4)-(6) we use ratio estimation.\\
There are three comments to the exercise:
\begin{itemize}
\item With sample plan is meant collection of all probabilities $p(s)$ for all possible samples s, i.e. sampling plan indicates all probabilities $p(s)$.
\item With standard error (SE) of estimator is meant square root of variance and not as usual, the estimated standard error.
\item With mean squared error (MSE) of an estimator $\hat{t}$ which is not biased for total t, is meant $E(\hat{t} - t)^2$. MSE can be calculated as sum of variance and square of the bias: $MSE = Var(\hat{t})+[E(\hat{t} - t)]^2$
\end{itemize}
\begin{enumerate}
\item \textbf{Sampling plan 1.} Company 4 should be included and one more company is sampled from 1,2,3 with probabilities proportional to number of employees:
\begin{itemize}
\item company 1: 0.2
\item company 2: 0.3
\item company 3: 0.5
\end{itemize}
Write down the sampling plan. Calculate Horvitz-Thompson(HT) estimator and show that it is unbiased. Calculate standard error(SE) of the estimator.\\
\fcolorbox{black}{ForestGreen!20}{
\begin{minipage}[t]{0.97\linewidth}
\textbf{Solution:}
Probabilities for samples in the sample plan 1: \\
$p(\{1,4\}) = 0.2 \quad p(\{2,4\}) = 0.3 \quad p(\{3,4\}) = 0.5$\\$\quad p(s) = 0$ for all other cases. \\

HT estimator where $y_i$ - turnover for company i can be obtained as:\\
$$\hat{t}_{HT} = \sum_{i \in  s}y_i/\pi_i$$
For $s_1 = \{1,4\}$ we get $\hat{t}_{HT}^{s_1} = y_1/\pi_1 + y_4 = 100/0.2 + 1000 = 1500$\\
For $s_2 = \{2,4\}$ we get $\hat{t}_{HT}^{s_2} = y_2/\pi_2 + y_4 = 200/0.3 + 1000 = 1666,67$\\
For $s_3 = \{3,4\}$ we get $\hat{t}_{HT}^{s_3} = y_3/\pi_3 + y_4 = 300/0.5 + 1000 = 1600$\\

$E(\hat{t}_{HT}) = 1500*0.2 + 1666.67*0.3 + 1600*0.5 = 300 + 500 + 800 = 1600 = t \implies$ the estimator is unbiased.\\
$V(\hat{t}_{HT}) = (1500 - 1600)^2 0.2+(1666.67 - 1600)^2 0.3+(1600-1600)^2 0.5) = 3333.33$\\
$SE = \sqrt{3333.33} = 57.7$

\end{minipage}}
\item \textbf{Sampling plan 2.} Company 4 should be included and one more company is sampled from 1,2,3 with probabilities:
\begin{itemize}
\item company 1: 0.5
\item company 2: 0.3
\item company 3: 0.2
\end{itemize}
Write down the sampling plan. Calculate HT estimator and show that it is unbiased. Calculate SE of the estimator. If SE will be much larger than in (1), find an estimator without using x which will be more accurate. Calculate bias, SE and $\sqrt{MSE}$ for it. \\
\fcolorbox{black}{ForestGreen!20}{
\begin{minipage}[t]{0.97\linewidth}
\textbf{Solution:}
Probabilities for samples in the sample plan 1: \\
$p(\{1,4\}) = 0.5 \quad p(\{2,4\}) = 0.3 \quad p(\{3,4\}) = 0.2$\\$\quad p(s) = 0$ for all other cases. \\

We use the same formula for HT estimator as in (1).\\
$\hat{t}_{HT}^{s_1} = y_1/\pi_1 + y_4 = 100/0.5 + 1000 = 1200$\\
$\hat{t}_{HT}^{s_2} = y_2/\pi_2 + y_4 = 200/0.3 + 1000 = 1666,67$\\
$\hat{t}_{HT}^{s_3} = y_3/\pi_3 + y_4 = 300/0.2 + 1000 = 2500$\\

$E(\hat{t}_{HT}) = 1200*0.5 + 1666.67*0.3 + 2500*0.2 = 600 + 
500 + 500 = 1600 = t$
$V(\hat{t}_{HT}) = (1200 - 1600)^2 0.5+(1666.67 - 1600)^2 0.3+(2500-1600)^2 0.2) = 243333.33$\\
$SE = \sqrt{243333.33} = 493.3, $ what is much larger than in (1), thus, we need to find a new estimator without using x.\\

We have two strata: a stratum with full count of units (company 4) and a stratum where sampling is taking place. We can calculate an estimate for the later stratum based on a sample and summarize it with the total from the earlier stratum. In this case the estimate for stratum $s = \{i,4\}$ will be calculated as $\hat{t} = 3y_i + y_4$.
Then the values for ratio estimators and probabilities are:\\
\end{minipage}}\\
\fcolorbox{black}{ForestGreen!20}{
\begin{minipage}[t]{0.97\linewidth}
\begin{center}
\begin{tabular}{r|rrr}
s& $\{1,4\}$ & $\{2,4\}$ & $\{3,4\}$\\ 
\hline
$\hat{t}_R$ & 1300 & 1600 & 1900\\
\hline
p(s) & 0.5 & 0.3 & 0.2\\
\hline
\end{tabular}
\end{center}
$E(\hat{t}_R) = 1300*0.5 + 1600*0.3 + 1900*0.2 = 1510$\\
$V(\hat{t}_R) = (1300-1510)^2*0.5 + (1600-1510)^2*0.3 + (1900-1510)^2*0.2 = 54900$\\
$SE = 234.3$\\
$MSE = 549000 + 90^2 = 63000$ and $RMSE = \sqrt{MSE} = 251.0$
\end{minipage}}
\item \textbf{Sampling plan 3 (SRS).} We sample an SRS of 2 companies. Write down the sampling plan. Calculate HT estimator and show that it is unbiased. Calculate SE of the estimator. \\
\fcolorbox{black}{ForestGreen!20}{
\begin{minipage}[t]{0.97\linewidth}
\textbf{Solution:}
There are possible 6 different samples of 2 companies from 4: $\{i,j\} = \{1,2\}, \{1,3\}, \{1,4\}, \{2,3\}, \{2,4\}$ and $\{3,4\}$. Each sample has a probability $p(\{i,j\})=1/6$. \\

All sampling probabilities are $n/N = 1/2$, and then $$\hat{t}_{HT} = \sum_{i \in s}y_/\pi_i=2\sum_{i \in s}y_i=4\bar{y}_s$$
$E(\hat{t}_{HT}) = 1/6*2 (100*3+200*3+400*3+1000*3) = 1600$\\

$V(4\bar{y}_s)=E(4\bar{y}_s-1600)^2 = \sum_{s:|s|=2}(4\bar{y}_s-1600)^2\frac{1}{6} = $\\
  $= \frac{1}{6}\big[(600-1600)^2+(800-1600)^2+(2200-1600)^2+(1000-1600)^2+$\\
  $+(2400-1600)^2+(2600-1600)^2\big] = 666666.667$\\

$\implies SE = 816.5$\\

NO: For this case we could also calculate SE based on formula for the variance:\\
$V(N\bar{y}_s) = N^2\frac{s^2}{n}(1-\frac{n}{N})$, where $s^2 = \frac{1}{N-1}\sum_{i=1}^N(y_i-\bar{y})^2$\\

By using the formula we get: $s^2 = (300^2+200^2+100^2+600^2)/3=500000/3$ and \\
$V(4\bar{y}_s)=4^2\frac{500000/3}{2}(1-\frac{1}{2})=4*500000/3 = 666666.667.$
\end{minipage}}
\item Find an expectation of ratio estimator for sampling plan 1 (1). Calculate SE of the ratio estimator. Calculate MSE and $\sqrt{MSE}$.\\
\fcolorbox{black}{ForestGreen!20}{
\begin{minipage}[t]{0.97\linewidth}
\textbf{Solution:}
The total number of employees is $20+30+50+200=300$, then ratio estimation is $$\hat{t}_R=300\frac{\sum_s y_i}{\sum_s x_i}.$$ 
The values for ratio estimators and probabilities are:
\begin{center}
\begin{tabular}{r|rrr}
s& $\{1,4\}$ & $\{2,4\}$ & $\{3,4\}$\\ 
\hline
$\hat{t}_R$ & 1500 & 1565.22 & 1560\\
\hline
p(s) & 0.2 & 0.3 & 0.5\\
\hline
\end{tabular}
\end{center}
Expectation:\\
$E(\hat{t}_R) = 1500*0.2 + 1565.22*0.2 + 1560*0.5 = 300 + 469.6 + 780 = 1549.6$\\
$V(\hat{t}_R) = E(\hat{t}_R - 1549.6)^2 = 49.6^2*0.2 + 15.62^2*0.3+10.4^2*0.5 = 619.31$\\
$SE = \sqrt{619.31} = 24.9$\\
$MSE = 619.31 + 0.4^2 = 3159.5$ and $RMSE = \sqrt{MSE} = \sqrt{3159.5} = 56.2$\\
RMSE can be compared with SE for unbiased estimators.
\end{minipage}}

\item Find and expectation of ratio estimator for sampling plan 2 (2). Calculate SE of the ration estimator. Calculate MSE and $\sqrt{MSE}$.\\
\fcolorbox{black}{ForestGreen!20}{
\begin{minipage}[t]{0.97\linewidth}
\textbf{Solution:}

The values for ratio estimators and probabilities are:
\begin{center}
\begin{tabular}{r|rrr}
s& $\{1,4\}$ & $\{2,4\}$ & $\{3,4\}$\\ 
\hline
$\hat{t}_R$ & 1500 & 1565.22 & 1560\\
\hline
p(s) & 0.5 & 0.3 & 0.2\\
\hline
\end{tabular}
\end{center}
$E(\hat{t}_R) = 1531.6 \quad V(\hat{t}_R) = 999.68 \quad SE = 31.6$\\
$MSE = 5678.24 \quad RMSE = 75.4$
\end{minipage}}

\item Find and expectation of ratio estimator for sampling plan 3 (3). Calculate SE of the ratio estimator. Calculate MSE and $\sqrt{MSE}$.\\
\fcolorbox{black}{ForestGreen!20}{
\begin{minipage}[t]{0.97\linewidth}
\textbf{Solution:}

The values for ratio estimators, each with probability 1/6 are:
\begin{center}
\begin{tabular}{r|rrrrrr}
s& $\{1,2\}$ & $\{1,3\}$ & $\{1,4\}$ & $\{2,3\}$ & $\{2,4\}$ & $\{3,4\}$\\ 
\hline
$\hat{t}_R$ & 1800 & 1714.29 & 1500 & 1875 & 1565.22 & 1560\\
\end{tabular}
\end{center}
$E(\hat{t}_R) = 1669.1 \quad V(\hat{t}_R) = 18810.07 \quad SE = 137.2$\\
$MSE = 23584.88 \quad RMSE = 153.6$

\end{minipage}}
\item \textbf{Compare sampling plans and estimators.} If we do not have available information on number of employees, which sampling plan (1)-(3) would you choose? Which sampling plan and estimator would you choose if there is available information on number of employees? \\
\fcolorbox{black}{ForestGreen!20}{
\begin{minipage}[t]{0.97\linewidth}
\textbf{Solution:}
If we do not have available information on number of employees:\\
All HT estimator were unbiased with the standard errors:
\begin{itemize}
\item Utvalgsplan 1: 57.7
\item Utvalgsplan 2: 493.3 (Alternativ estimator: SE = 234.2 with bias = -90 and RMSE = 251.0)
\item Utvalgsplan 3: 816.5
\end{itemize}
A superior sample plan is (1): since has a higher sampling probability for bigger companies and the biggest company is necessary included.
\\
If we do have available information on number of employees:
All estimators and standard error are:
\begin{itemize}
\item Utvalgsplan 1 and HT-estimator: SE = 57.7, bias = 0
\item Utvalgsplan 2 and HT-estimator: SE = 493.3, bias = 0 (Alternativ estimator: SE = 234.2 with bias = -90 and RMSE = 251.0)
\item Utvalgsplan 3 and HT-estimator: SE = 816.5, bias = 0
\item Utvalgsplan 1 and ratio estimation: SE = 24.9 and bias = -50.4; RMSE = 56.2
\item Utvalgsplan 2 and ratio estimation: SE = 31.6 and bias = -68.4; RMSE = 75.4
\item Utvalgsplan 3 and ratio estimation: SE = 137.2 and bias = +69.1; RMSE = 153.6
\end{itemize}
Looking at the estimator we can make the next conclusions:
\begin{itemize}
\item Firstly, we can notice that HT-estimator is working poorly for the cases when the sampling probability is negatively correlated with y values (See at (2)). In (2) SE is 8,5 times larger than SE in (1). Ratio estimator is much more robust. RMSE of sampling plan 2 (See at (5)) is just 1.3 times larger than (1).
\item It is not reasonable to take a SRS in companies' surveys when there is a big variation of y values.
\item It will be unlucky if we would choose as sampling plan with an estimator defined in (2) not looking at the fact that it is unbiased. RMSE for ratio estimation  in (5) which is just 15\% of SE in (2). If we do not have available additional information x if would be better to use as an estimator $\hat{t} = 3y_1 + y_4$, not looking at the fact that it is biased.
\item It is not so easy to conclude here which estimator to choose. Sampling plan 1 is still a superior of the three samplings plans. Its ratio estimation (4) has a much lower SE than it's HT estimation (1), but at the same moment the bias in  (4) is so big that RMSE in (4) is almost the same as SE in (1). For populations and samples with usual sizes, a bias in ratio estimation will be considerably smaller so that the ratio estimator will be preferred.  
\end{itemize}



\end{minipage}}
\end{enumerate}

\section*{Exercise 2}
\textbf{\color{ForestGreen}(R code available)} The file azcounties.dat gives data from the 2000 U.S. Census on population and housing unit counts for the counties in Arizona (excluding Maricopa County and
Pima County, which are much larger than the other counties and would be placed in a separate stratum). For this exercise, suppose that year 2000 population ($M_i$) is known and you want to take a sample of counties to estimate the total number of housing units ($t =\sum_{i \in U} t_i$). The file has the value of $y_i$ for every county so you can calculate the population total and variance.
\begin{enumerate}
\item Calculate the selection probabilities $\psi_i$ for a sample of size $1$ with probability
proportional to 2000 population. Find $\hat{t}_\psi$ for each possible sample, and calculate
the theoretical variance $V(\hat{t}_\psi)$.\\
\fcolorbox{black}{ForestGreen!20}{
\begin{minipage}[t]{0.97\linewidth}
\textbf{Solution:}
The 2000 U.S. Census on population and housing unit counts for the counties in Arizona are provided. We have $n=1$, $N=13$ and $t=\sum_{i \in U}t_i=572\,221$.\\
Select all possible samples with unequal-probabilities $\psi_i \propto M_i$, where the $M_i$ are 2000 population. Find $\hat{t}_\psi$ for each sample, and calculate the theoretical variance $V(\hat{t}_\psi)$.
$$\hat{t}_\psi=\sum_{i \in S}\frac{t_i}{\psi_i},\quad\psi_i=\frac{M_i}{\sum_{i=1}^{13}M_i},$$
$$V(\hat{t}_\psi)=E[(\hat{t}_\psi-t)^2]=\sum_{samples\,i}\psi_i(\hat{t}_\psi-t)^2\cdot$$
\small
\begin{center}
\begin{tabular}{rrrrr}
County & $t_i$ & $\psi_i$ & $\hat{t}_\psi$ & $\psi_i(\hat{t}_\psi-t)^2$\\ 
\hline
1& 31 621& 0.0572& 553 292.1& 20 477 223\\
2& 51 126 &0.0969 &527 405.6& 194 693 778\\
3&  53 443& 0.0958 &558 108.6& 19 071 085 \\
4& 28 189& 0.0423 &667 034.6 &379 902 891\\
5&  11 430& 0.0276& 414 597.1& 684 957 758\\
6 & 3 744 &0.0070 &532 113.6 &11 318 255 \\
7 & 15 133& 0.0162& 932 417.7& 2 105 687 838\\
8&  80 062 &0.1276& 627 317.4& 387 423 351\\
9 & 47 413 &0.0802& 590 892.8& 27 974 547\\
10& 81 154 &0.1480& 548 502.8& 83 232 656\\
11 &13 036 &0.0316& 412 582.0& 805 214 838\\
12 &81 730 &0.1379& 592 659.0& 57 604 012\\
13& 74 140 &0.1317& 562 787.3& 11 723 897\\
\hline
Sum & 572 221& 1.0000& & 4 789 282 131 \\
\end{tabular}
\end{center}

\end{minipage}}
\item Repeat (1) for an equal probability sample of size $1$. How do the variances compare?
Why do you think one design is more efficient than the other? \\
\fcolorbox{black}{ForestGreen!20}{
\begin{minipage}[t]{0.97\linewidth}
\textbf{Solution:}
Now select an SRS of size $n=1$ and repeat the calculations in (1).\\
Under SRS with $n=1$, we have $\psi_i=1/13$ for each county. Each possible sample has also the same probability of selection.\\

\small
\begin{center}
\begin{tabular}{rrrrr}
County & $t_i$ & $\psi_i$ & $\hat{t}_{srs}$ & $(\hat{t}_{srs}-t)^2/13$  \\ 
\hline
1&  31 621& 0.0769& 411 073& 1 997 590 608\\
2&  51 126& 0.0769& 664 638& 656 992 453\\
3&  53 443& 0.0769& 694 759& 1 155 043 188\\
4&  28 189& 0.0769& 366 457& 3 256 832 592\\
5&  11 430& 0.0769&  14 859& 13 804 863 397\\
6&   3 744& 0.0769&  48 672& 21 084 888 877\\
7&  15 133& 0.0769& 196 729& 10 845 710 928\\
8&  80 062& 0.0769& 1 040 806& 16 890 146 325\\
9&  47 413& 0.0769&  616 369& 149 926 608\\
10& 81 154& 0.0769&1 055 002& 17 929 037 997\\
11& 13 036& 0.0769& 169 468& 12 477 690 693\\
12& 81 730& 0.0769&  1 062 490& 18 489 514 797\\
13& 74 140& 0.0769& 963 820& 11 796 136 677\\
\hline
Sum & 572 221& 1.0000&  &1.30534$\times 10^{11}$ \\
\end{tabular}
\end{center}

The design with unequal-probabilities is more efficient than the SRS. Because, population and housing unit counts are highly correlated. We have $\rho=0.9905$. This leads to that the values of $\hat{t}_\psi=t_i/\psi_i$ do not vary so much as those under SRS from sample to sample. 
\end{minipage}}
\end{enumerate}

\section*{Exercise 3}
\textbf{\color{ForestGreen}(R code available)} Let’s return to the situation in Exercise 3 of Session 2, in which we took an SRS to estimate the average and total numbers of refereed publications of faculty and research
associates. Now, consider a probability proportional to size (pps) sample of faculty. The $27$ academic units range in size from $2$ to $92$. We used Lahiri’s method to choose $10$ (primary sampling units)psus with probabilities proportional to size and with replacement, and took an SRS of four (or fewer, if $M_i < 4$) members from each psu. Note that academic unit $14$ appears three times in the sample; each time it appears, a different subsample was collected.
\begin{center}
\begin{tabular}{rrcl}
Academic & &  &  \\
unit & $M_i$ & $\psi_i$ & \multicolumn{1}{c}{$y_{ij}$}  \\
\hline
14 &65& 0.0805452 &3, 0, 0, 4 \\
23& 25 &0.0309789& 2, 1, 2, 0 \\
9& 48& 0.0594796& 0, 0, 1, 0 \\
14 &65& 0.0805452& 2, 0, 1, 0\\
16& 2 &0.0024783 &2, 0\\
6 &62 &0.0768278 &0, 2, 2, 5\\
14& 65& 0.0805452 &1, 0, 0, 3\\
19 &62& 0.0768278& 4, 1, 0, 0\\
21& 61& 0.0755886& 2, 2, 3, 1\\
11& 41& 0.0508055& 2, 5, 12, 3\\
\end{tabular}
\end{center}
Find the estimated total number of publications, along with its standard error. \\
\fcolorbox{black}{ForestGreen!20}{
\begin{minipage}[t]{0.97\linewidth}
\textbf{Solution:}
Academic units are selected with a pps sampling with replacement. $N=27$ and $n=10$. Find $\hat{t}$ and its standard error.
$$\hat{t}=\frac{1}{R}\sum_{i \in R} \frac{\hat{t}_i}{\psi_i},\quad \hat{t}_i=\sum_{j \in R_i}\frac{M_i}{m_i}y_{ij},$$
$$\hat{V}(\hat{t})=\frac{1}{n}\frac{1}{n-1}\sum_{i \in R}\Big(\frac{\hat{t}_i}{\psi_i}-\hat{t}\Big)^2,$$
where $R$ denotes the set of $n$ units in the sample, including the repeats.\\
$\hat{t}=1\,371.90$,\\
$\widehat{SE}(\hat{t})=1\,179.47/\sqrt{10}=372.98$.
\small
\begin{center}
\begin{tabular}{rrclrr}
psu & $M_i$ & $\psi_i$ & \multicolumn{1}{c}{$y_{ij}$} &\multicolumn{1}{c}{$\hat{t}_i$} &  \multicolumn{1}{c}{$\hat{t}_i/\psi_i$}\\
\hline       
14 &65& 0.0805452 & 3, 0, 0, 4 & 113.75& 1 412.25\\
23& 25 &0.0309789 & 2, 1, 2, 0 &31.25& 1 008.75\\
9& 48& 0.0594796 & 0, 0, 1, 0 &12.00& 201.75\\ 
14 &65& 0.0805452 & 2, 0, 1, 0& 48.75& 605.25\\
16& 2 &0.0024783 &2, 0 &2.00& 807.00\\
6 &62 &0.0768278 &0, 2, 2, 5& 139.50& 1 815.75\\
14& 65& 0.0805452 &1, 0, 0, 3& 65.00&807.00 \\
19 &62& 0.0768278 & 4, 1, 0, 0& 77.50& 1 008.75\\
21& 61& 0.0755886 & 2, 2, 3, 1 &122.00& 1 614.00\\
11& 41& 0.0508055 & 2, 5, 12, 3 &225.50&4 438.50 \\
\hline
$Mean$&&&&&1 371.90\\
$SD$ &&&&& 1 179.47 \\
\end{tabular}
\end{center}

\end{minipage}}

\section*{Exercise 4}
\textbf{\color{ForestGreen}(R code available)} A two-stage unequal probability sample without replacement of size $n=5$ from the population of statistics classes of size $N=15$ (see Lohr, 2019, pp.247-248) is taken. The psu inclusion probabilities
are proportional to the class sizes $M_i$. We have $M_0=\sum_{i \in U}M_i=647$. The data are in file classpps.dat. A sample of $m_i=4$ ssus is selected with a simple random sampling without replacement from each sample class. The total number of hours spent studying statistics is of interest. Here,
the sampling fraction $n/N$ is $1/3$, so the with-replacement variance is likely to
overestimate the without-replacement variance. The joint inclusion probabilities for
the psus are given in file classppsjp.dat.
\begin{enumerate}
\item Calculate $\hat{V}_{HT}(\hat{t}_{HT})$ and $\hat{V}_{SYG}(\hat{t}_{HT})$ for this dataset.\\
\fcolorbox{black}{ForestGreen!20}{
\begin{minipage}[t]{0.97\linewidth}
\textbf{Solution:}
A two-stage sample with unequal probabilities WOR. $N=15$, $n=5$, $m_i=4$, and $M_0=647$. We have $\pi_i\propto M_i$, and $\pi_{j|i}=4/M_i$, for $j\in S_i$. An SRSWOR is used to select the SSUs.\\
$$\hat{t}_{HT}=\sum_{i \in S}\frac{\hat{t}_i}{\pi_i},\quad \hat{t}_i=\sum_{j\in S_i}\frac{y_{ij}}{\pi_{j|i}}\cdot$$
$$\hat{V}(\hat{t}_{HT})=\hat{V}_{psu}+\hat{V}_{ssu} = \hat{V}_{psu}+\sum_{i \in S}\frac{\hat{V}(\hat{t}_i)}{\pi_i}$$
$$\hat{V}(\hat{t}_i)=M_i^2\Big(1-\frac{4}{M_i}\Big)\frac{s_i^2}{4},\quad s_i^2=\frac{1}{3}\sum_{j\in S_i}(y_{ij}-\bar{y}_i)^2\cdot$$
\begin{center}
\begin{tabular}{rrrrrr}
class & $\pi_i$ &$M_i$ & $\hat{t}_i$ & $s_i^2$ & $\hat{V}(\hat{t}_i)$\\
\hline
  4   &     0.17002&      22&   110.00& 0.1666667&   16.5000\\
  10  &     0.26275&      34&   106.25& 0.7291667&  185.9375\\
   1  &     0.34003&      44&   154.00& 1.6666667&  733.3333\\
   9  &     0.41731&      54&   195.75& 4.2291667& 2 854.6875\\
  14  &     0.77280&     100&   200.00& 0.5000000& 1 200.0000\\
\end{tabular}
\end{center}
$$\sum_{i \in S}\frac{\hat{V}(\hat{t}_i)}{\pi_i}=\frac{16.5}{0.17002}+\frac{185.9}{0.26275}+\cdots+\frac{1\,200}{0.77280}=11\,354.9\cdot$$
\vspace{-0.5 cm}
\begin{itemize}
\item The first term of $\hat{V}(\hat{t}_{HT})$ can be calculated by using either the Horvitz-Thompson (HT) estimator or the Sen-Yates-Grundy (SYG) estimator.
\begin{eqnarray*}
\hat{V}_{psu;HT}=\sum_{i \in S}\sum_{k\in S}\frac{(\pi_{ik}-\pi_i\pi_k)}{\pi_{ik}}\frac{\hat{t}_i}{\pi_i}\frac{\hat{t}_k}{\pi_k}=6\,059.6\cdot
\end{eqnarray*}
\vspace{-0.5cm}
\begin{eqnarray*}
\hat{V}_{psu;SYG}=-\frac{1}{2}\sum_{i \in S}\sum_{k\neq i \in S}\frac{(\pi_{ik}-\pi_i\pi_k)}{\pi_{ik}}\Big(\frac{\hat{t}_i}{\pi_i}-\frac{\hat{t}_k}{\pi_k}\Big)^2=54\,784.5\cdot
\end{eqnarray*}
\end{itemize}
\begin{itemize}
\item $\hat{V}_{HT}(\hat{t}_{HT})=6\,059.6+11\,354.9=17\,414.46$
\item $\hat{V}_{SYG}(\hat{t}_{HT})=54\,784.5+11\,354.9=66\,139.41$
\end{itemize}
$\implies$ Because of the small sample size, i.e. $n=5$, these estimators are quite unstable. This is the reason of the big difference between the two estimates. 
\end{minipage}}

\item SAS software approximates the without-replacement variance in unequal-probability
sampling using $$\Big(1-\frac{n}{N}\Big)\hat{V}_{WR}(\hat{t}_{HT})\cdot$$
Calculate this approximation for the class data.
\\
\fcolorbox{black}{ForestGreen!20}{
\begin{minipage}[t]{0.97\linewidth}
\textbf{Solution:}
Use the fpc-adjusted with-replacement variance estimator to estimate $V(\hat{t}_{HT})$.
$$\hat{t}_{HT}=\sum_{i \in S}\frac{\hat{t}_i}{\pi_i}=\frac{110}{0.17002}+\frac{106.25}{0.26275}+\cdots+\frac{200}{0.77280}=2\,232.15\cdot$$
\begin{eqnarray*}
\hat{V}(\hat{t}_{HT})&=&(1-\frac{n}{N})\hat{V}_{WR}(\hat{t}_{HT})\\
&=&\Big(1-\frac{n}{N}\Big)\frac{n}{n-1}\sum_{i \in S}\Big(\frac{\hat{t}_i}{\pi_i}-\frac{\hat{t}_{HT}}{n}\Big)^2\\
&=&\Big(1-\frac{5}{15}\Big)\frac{5}{4}\Bigg[\Big(\frac{110}{0.17002}-\frac{2\,232.15}{5}\Big)^2+\cdots\Bigg]\\
&=&\frac{2}{3}(97\,187.37)=64\,791.58
\end{eqnarray*}

\end{minipage}}
\item How do these estimates compare, and how do they compare with the with replacement
variance for $\hat{t}_{HT}$? \\
\fcolorbox{black}{ForestGreen!20}{
\begin{minipage}[t]{0.97\linewidth}
\textbf{Solution:}
Comparison of the estimates.
\begin{center}
\begin{tabular}{lcrr}
variance estimator & \vline & estimate of $V(\hat{t}_{HT})$ & standard error\\
\hline
$\hat{V}_{HT}(\hat{t}_{HT})$ & \vline & 17 414.46 &131.96\\
$\hat{V}_{SYG}(\hat{t}_{HT})$ & \vline & 66 139.41 &257.18\\
$\hat{V}_{WR}(\hat{t}_{HT})$ & \vline &  97 187.37&311.75\\
$(1-n/N)\hat{V}_{WR}(\hat{t}_{HT})$ & \vline & 64 791.58&254.54 \\
\end{tabular}
\end{center}
\begin{itemize}
\item $\hat{V}_{HT}(\hat{t}_{HT})<(1-n/N)\hat{V}_{WR}(\hat{t}_{HT})<\hat{V}_{SYG}(\hat{t}_{HT})<\hat{V}_{WR}(\hat{t}_{HT})$
\item Here, we have $\widehat{SE}_{SYG}(\hat{t}_{HT})\approx (\sqrt{1-n/N})\,\widehat{SE}_{WR}(\hat{t}_{HT})$
\end{itemize}
\end{minipage}}
\end{enumerate}

\end{document}