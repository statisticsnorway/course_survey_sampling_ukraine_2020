\documentclass[12pt]{article}

% Packages
\usepackage[margin=0.3in]{geometry}
\usepackage{graphicx}
\usepackage{amsmath}
\usepackage{float}
\usepackage[dvipsnames]{xcolor}
\usepackage{etoolbox}
\usepackage[russian]{babel}

\pagestyle{plain}


\begin{document}

\title{Translation of some terms}
\author{Badina, Svetlana \& O\u{g}uz--Alper, Melike \& Pekarskaya, Tatsiana, Statistics Norway}
\maketitle
\small
\begin{center}
\begin{tabular}{l|l}
\hline

Accuracy & Точность (с точки зрения смещения).\\ & Правильность (Аккуратность) \\
Asymptotically & Асимптотически \\
Asymptotically unbiased & Асимптотически несмещенный\\
Auxiliary variable & Вспомогательная переменная\\
Bias & Смещение   \\
Biased estimator & Смещенная оценка\\
Calibrated (adjusted) weights & Откалибрированные (скорректированные) веса\\
Calibration factors & Коэффициенты калибровки\\
Cell(class) - mean imputation & Импутация (вставка, вменение) среднего ячейки (класса)\\
CLT: central limited theorem & Центральная предельная теорема \\
Clusters & Кластеры\\
Cold-deck & "Холодная колода (дека)". Метод импутации данных, \\ & при котором недостающее  значение заменяется из \\ &(раннее полученных) исторических данных \\
Composition & Композиция\\
Confidence interval & Доверительный интервал \\
Consistent & Согласующийся\\
Deductive imputation & Дедуктивная импутация (вставка, вменкеие) данных\\
Design (sampling) weights $d_i = \pi_i^{-1}$ & Выборочные веса \\
Design effect (DEFF) & Эффект дизайна/конструирования. Эффект схемы\\
Deterministic imputation & Детерминированная импутация (вставка, вменение) данных\\
Efficient estimator & Эффективная оценка \\
Equal Probabilities SElection Method (EPSEM) & Метод выбора равных вероятностей \\
Estimated sampling variance & Оценка выборочной дисперсии \\
Estimator & Оценочная функция, оценочное правило, функция от \\&выборки.  В некоторых источниках переводится как \\&Статистическая оценка, но тогда надо понимать, что\\& подразумевается функция или правило, которое \\& используется для оценивания параметров\\
Evenly scattered & Равномерно разбросанны\\
Expansion estimator == Horvitz–Thompson estimator & Оценочная величина Горвица-Томпсона\\
Explicit model & Явная модель\\
Finite population correction (fpc) & Поправка на конечность совокупности\\
First-order Taylor series & Ряд Тейлора с отсечения всех членов выше первого порядка\\
Homogeneity & Однородность\\
Hot-deck & ``Горячая колода'', Метод импутации данных, при котором \\ & недостающее(ие) значение(я) заменяется(ются) значением из \\ & только что полученных данных\\
Incentives and disincentives & Поощряющие и сдерживающие факторы\\
Inclusion probability $\pi_i$& Вероятность попадания(включения) единицы i в выборку \\
independent and identically distributed (iid) & Независимый и равномерно распределенный\\
Invariance & Инвариантность\\

\hline
\end{tabular}
\end{center}

\begin{center}
\begin{tabular}{l|l}
\hline
Item nonresponse & Отсутствие ответа по какием-то вопросам \\&(респондент ответил на чать вопросов) \\
Jackknife method & Метода «складного ножа» (или jackknife) \\
Jackknife variance estimator & Оценочная величина дисперсии при помощи метода "складного ножа"\\
Linear combination & Линейная комбинация\\
MAR, missing at random & Отсутствие случайно\\
Margin-of-Error (MoE) & Предельная ошибка выборки\\
MCAR, Missing completely at random & Отсутствие полностью случайно\\
Mean squared error & Среднеквадратическая ошибка \\
Mode of data-collection & Способо сбора данных\\
Multiple imputation & Множественная вставка\\
Multi-stage cluster sampling(MCS) & Многоступенчатая выборка\\
Mutually exclusive and exhaustive& Взаимоисключающий и исчерпывающий\\ 
Naive bootstrap & Метод наивного бутстре(а,э)па\\
Nearest-Neighbour& Метод ближайшего соседа\\
Objective inference framework & Руководство по статистическому оцениванию\\
Observation unit & Наблюдаемая единица\\
One-stage cluster sampling(SCS) & Одноступенчатая выборка\\
Optimal allocation & Оптимальное распределение выборки\\
Partial derivatives & Частная производная\\
Population mean/variance & Среднее/дисперсия генеральной совокупности\\ 
Post-stratified & Калибровка выборки, или пост-стратификация\\
PPS (probability proportional to size) & Вероятность пропорциональная размеру\\
Precision & Точность (с точки зрения дисперсии). Прецизионность\\
Preserve & Сохранять\\
Primary sampling units(PSU)&Первичные наблюдаемые единицы выборки. \\ & Единицы отбора первой ступени \\
Probability of selecting s $p(s)$& Вероятность выбрать выборку s \\
Proportional allocation & Пропорциональное распределение выборки\\
Raking weight adjustment factor& Коэффициент корректировки весов методом ``рейкинг"\\
Raking-ratio adjustment & Корректировка методом ``рейкинг”\\
Random imputation & Случайнаяимпутация (вставка, вменение) данных\\
Ratio-estimation & Оценивание при помощи(на основе) соотношений\\
Reference sampling strategy & Рекомендация/эталон для стратегии выборочного \\& наблюдения. Cтратегия эталонной выборки\\
Regression estimation & оценивание при помощи регрессии\\
Replicate & Репдикация, повтор\\
Representative sample & Репрезентативная выборка\\
Rescaling bootstrap method & Масштабирование метода бутстре(а)па\\
Response burden & Нагрузка в связи с опросом, бремя в связи с \\ & необходимостью ответа на опрос\\
Response propensity(probability) & Склонность к ответу (вероятность ответа)\\
Response rate & Процент ответа\\
Responsive survey design & Опросник, на который легко и возможно ответить\\
Sample mean / variance & Среднее значение/дисперсия выборки \\
Sample s& Выборка, выборочная совокупность\\
Sample size & Размер выборки\\
Sample space $\Omega$& Пространство, включающее все возможные выборки; \\& Множество всех вохможных выборок\\
Sampling design & Стратегия формировани/конструирования выборки (Дизайн выборки)\\
Sampling distribution & Выборочное распределение \\
Sampling fraction & Доля выборки \\
Sampling frame & Основа выборки\\
Sampling strategy & Стратегия выборочного наблюдения\\
Sampling unit & Единица выборки\\
Sampling variance & Выборочная дисперсия \\
\hline
\end{tabular}
\end{center}

\begin{center}
\begin{tabular}{l|l}
\hline
Secondary sampling units (SSU)& Вторичные наблюдаемые единицы выборки. \\ & Единицы отбора вторичной ступени (вторичные единицы)\\
Selection bias & Ошибка отбора\\
Simple random sample & Простая случайная выборка \\
Skewness & Коэффициент ассиметрии\\
Smooth function & Сглаженная функция\\
SRSWOR: simple random sample without replacement & Простая случайная выборка без возмещения \\
SRSWR: simple random sample with replacement & Простая случайная выборка с возмещением \\
Stochastic regression imputation & Импутация (вставка, вменение) при помощи стохастической \\& регрессии\\
Stratified sampling& Стратифицированная выборка\\
Stratum(a) & Страта(ы)\\
Sum $\sum_{i \in s}y_i$ & Сумма величины $y_i$ для всех единиц $i$ в выборке\\
Taylor series expansion & Разложение в ряд Тейлора\\
U finite population or universe& Конечная генеральная совокупность\\
Ultimately sampling units (USU)& Итоговые наблюдаемые единицы выборки. \\& Единицы отбора конечной ступени\\
Unbiased estimator & Несмещенная оценка \\
Unconditional & Безусловный \\
Unit nonresponse & Отсутствие ответа от респондента полностью \\&(мы не смогли получить никакого ответа от респондента)\\
Variance & Дисперсия \\
Variance decomposition & Декомпозиция дисперсии\\
Weight adjustment & Корректировка веса\\
Weighting class adjustment & корректировка класса весов \\


\hline
\end{tabular}
\end{center}

\end{document}