\documentclass[12pt]{article}

% Packages
\usepackage[margin=0.6in]{geometry}
\usepackage{graphicx}
\usepackage{amsmath}
\usepackage{float}
\usepackage[dvipsnames]{xcolor}
\usepackage{etoolbox}
\usepackage{soul}
%\usepackage{tcolorbox}
%\definecolor{block-gray}{gray}{0.85}

\pagestyle{plain}


\begin{document}

\title{Exercise ark \#4.\\ Cluster sampling (one-stage, two-stage)}
\author{O\u{g}uz--Alper, Melike \& Pekarskaya, Tatsiana, Statistics Norway}
\maketitle
\section*{Exercise 0}
Discuss:
\begin{itemize}
\item What is the difference between stratified and cluster samples?\\
\item What is the difference between one-stage and two-stage sampling?
\end{itemize}

\section*{Exercise 1}
Kleppel et al. (2004) report on a study of wetlands in upstate New York. Four wetlands
were selected for the study: Two of the wetlands drain watersheds from small towns
and the other two drain suburban watersheds. Quantities such as pH were measured
at two to four randomly selected sites within each of the four wetlands. 
\begin{enumerate}
\item Describe why this is a cluster sample. What are the psus? The ssus? How would you estimate the average pH in the suburban wetlands?
\item The authors used Student’s two-sample t test to compare the average pH from the sites in the suburban wetlands with the average pH from the sites in the small town wetlands, treating all sites as independent. Is this analysis appropriate? Why, or why not? \hfill (Lohr, 2019, p.208)
\end{enumerate}

\section*{Exercise 2}
 A city council of a small city wants to know the proportion of eligible voters that
oppose having a incinerator of Phoenix garbage opened just outside of the city limits.
They randomly select $100$ residential numbers from the city’s telephone book that contains $3\,000$ such numbers. Each selected residence is then called and asked for (a) the total number of eligible voters and (b) the number of voters opposed to the incinerator. A total of $157$ voters were surveyed; of these, $23$ refused to answer the question. Of the remaining $134$ voters, $112$ opposed the incinerator, so the council estimates the proportion by
$$\hat{p}=112/134=0.83582$$
with 
$$\hat{V}(\hat{p})=0.83582(1-0.83582)/134=0.00102$$
Are these estimates valid? Why, or why not? \hfill(Lohr, 2019, pp.207-208)


\section*{Exercise 3}
\textbf{\color{ForestGreen}(R code available)} The new candy Green Globules is being test-marketed in an area of upstate New York. The market research firm decided to sample $6$ cities from the $45$ cities in the area and  then to sample supermarkets within cities, wanting to know the number of cases of
Green Globules sold.
\begin{center}
\begin{tabular}{rrl}
 & Number of &  \\
City& supermarkets & Number of cases sold \\
\hline
1& 52& 146, 180, 251, 152, 72, 181, 171, 361, 73, 186 \\
2& 19& 99, 101, 52, 121\\
3 &37& 199, 179, 98, 63, 126, 87, 62\\
4 &39& 226, 129, 57, 46, 86, 43, 85, 165\\
5 &8 &12, 23\\
6 &14 &87, 43, 59\\
\end{tabular}
\end{center}
Obtain summary statistics for each cluster. Plot the data, and estimate the total number of cases sold, and the average number sold per supermarket, along with the standard errors of your estimates. \hfill (Lohr, 2019, p.209)

\section*{Exercise 4}
An accounting firm is interested in estimating the error rate in a compliance audit it is
conducting. The population contains $828$ claims, and the firm audits an SRS of $85$ of
those claims. In each of the $85$ sampled claims, $215$ fields are checked for errors. One
claim has errors in $4$ of the $215$ fields, $1$ claim has $3$ errors, $4$ claims have $2$ errors,
$22$ claims have $1$ error, and the remaining $57$ claims have no errors. (Data courtesy of
Fritz Scheuren.)
\begin{enumerate}
\item Treating the claims as psus and the observations for each field as ssus, estimate
the error rate, defined to be the average number of errors per field, along with the
standard error for your estimate.
\item Estimate (with standard error) the total number of errors in the $828$ claims.
\item Suppose that instead of taking a cluster sample, the firm had taken an SRS of $85\times 215=18\,275$ fields from the $178\,020$ fields in the population. If the estimated
error rate from the SRS had been the same as in (1), what would the estimated
variance $\hat{V}(\hat{p}_{srs})$ be? How does this compare with the estimated variance
from (1)? \hfill (Lohr, 2019, p.210)
\end{enumerate}

\section*{Exercise 5}
\textbf{\color{ForestGreen}(R code available)} The file measles.dat contains data consistent with that obtained in a survey of parents
whose children had not been immunized for measles during a recent campaign to
immunize all children between the ages of $11$ and $15$. During the campaign, $7\,633$ children
from the $46$ schools in the area were immunized; $9\,962$ children whose records
showed no previous immunization were not immunized. In a follow-up survey to
explore why the children had not been immunized during the campaign, Roberts
et al. (1995) sent questionnaires to the parents of a cluster sample of the $9\,962$ children.
Ten schools were randomly selected, then a sample of the $m_i$ nonimmunized
children from each school was selected and the parents of those children were sent a
questionnaire.
\begin{enumerate}
\item Estimate, separately for each school, the percentage of parents who returned a consent form (variable \emph{returnf}). For this exercise, treat the “no answer” responses (value 9) as not returned.
\item Using the number of respondents in school $i$ as $m_i$, construct the sampling weight for each observation.
\item Estimate the overall percentage of parents who received a consent form along with a $95\%$ CI.
\item How do your estimate and interval in part (3) compare with the results you would have obtained if you had ignored the clustering and analyzed the data as an SRS?
Find the ratio:
\begin{center}
\begin{tabular}{c}
estimated variance from (3) \\
\hline
estimated variance if the data were analyzed as an SRS  \\
\end{tabular}
\end{center}
What is the effect of clustering? \hfill (Lohr, 2019, p.212)
\end{enumerate}

\end{document}