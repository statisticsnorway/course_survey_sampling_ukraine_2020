\documentclass[12pt]{article}

% Packages
\usepackage[margin=0.6in]{geometry}
\usepackage{graphicx}
\usepackage{amsmath}
\usepackage{float}
\usepackage[dvipsnames]{xcolor}
\usepackage{etoolbox}
\usepackage{soul}
\usepackage{arydshln}
%\definecolor{block-gray}{gray}{0.85}

\pagestyle{plain}


\begin{document}

\title{Solution ark \#6.\\ Survey nonresponse}
\author{O\u{g}uz--Alper, Melike \& Pekarskaya, Tatsiana, Statistics Norway}
\maketitle

\section*{Exercise 1}
We look at the Norwegian election survey from 1993. The sample consists of 3000 persons. 11 callbacks were used. The sample of 3000 is assumed to be a random sample.  We shall use the data after two callbacks. The number of responses were 1403.
\begin{enumerate}
\item Of the 1403 persons, 1190 said they voted in the Parliament election 1993. Assume that the nonresponse is MCAR and compute an estimate and a 95\% confidence interval for the proportion of voters in the population. \\
\fcolorbox{black}{ForestGreen!20}{
\begin{minipage}[t]{0.97\linewidth}
\textbf{Solution:}
With MCAR, the response sample is a random sample \\
$\hat{p} = 1190 / 1403 = 0.848, \quad \widehat{SE}(\hat{p}) = \sqrt{\frac{\hat{p}(1-\hat{p})}{n-1}\frac{N-n}{N}} = 0.00959$\\
95\% confidence interval:$\hat{p}\pm1.96\widehat{SE}(\hat{p})=0.848 \pm0.019 = [0.829;0.867]$
\end{minipage}}\\
\item The true voting proportion was 0.755. Compare the estimate and confidence interval from (1) with the true proportion. What can you say about the MCAR assumption?\\
\fcolorbox{black}{ForestGreen!20}{
\begin{minipage}[t]{0.97\linewidth}
\textbf{Solution:}
The estimate and CI have large bias from the ``true" value. Nonresponse can be ignored if $\bar{y}_r$ (sample mean for the respondents) approximately unbiased for the population mean. It does not hold in our case, thus, the nonresponse is not MCAR.
\end{minipage}}\\
To try to correct the bias estimation in part (1) we shall poststratify according to voting participation in the Parliament election in 1989. We use 3 groups: 
\begin{itemize}
\item Group 1: participating in the 1989 election: $N_1$ = 2 510 669
\item Group 2:  not participating in the 1989 election: $N_2$ = 508 288
\item Group 3: new voters: $N_3$ = 241 000
\end{itemize}
In the response sample we have the following result, where y = 1 indicate voting in 1993 election and y = 0 indicate not voting in 1993.
\begin{center}
\begin{tabular}{l|cc|cc|cc}
Group & 1 & & 2 & & 3 &\\ 
\hdashline
y & 0 & 1 & 0 & 1 & 0 & 1 \\
\hline
\hline
\# persons & 132 & 1060 & 58 & 57 & 23 & 73\\
\hdashline
Total & 1192 & & 115 & & 96 & \\
\end{tabular}
\end{center}
\item Find the post-stratified estimate for voting proportion in 1993 and compare with the estimate in part (1) and the true value 0.755.\\
\fcolorbox{black}{ForestGreen!20}{
\begin{minipage}[t]{0.97\linewidth}
\textbf{Solution:}
$$\hat{p}_{pst} = \frac{1}{N}\hat{t}_{pst} = \sum_{h=1}^3\frac{N_h}{N}\bar{y}_{hR}$$
$\bar{y}_{1R} = 1060/1192=0.8893$\\
$\bar{y}_{2R} = 57/115 = 0.4957$\\
$\bar{y}_{3R} = 73/96 = 0.7604$\\
$\implies \hat{t}_{pst} = 2667953$ and $\hat{p}_{pst} = 2667953/3259957 = 0.818$\\
The bias was reduced compared to (1), however the estimator is still biased.
\end{minipage}}
\item Under which condition is post-stratified estimator approximately unbiased? Is that the case in (3)? If not, how would you cope with this problem? \\
\fcolorbox{black}{ForestGreen!20}{
\begin{minipage}[t]{0.97\linewidth}
\textbf{Solution:}
``The post-stratified estimator approximately unbiased if
within a post-stratum:
\begin{itemize}
\item the response variable $y_i$ constant
\item the response propensities $\phi_i$ the same for every unit
\item the response $y_i$ uncorrelated with the response propensity $\phi_i$"(Lecture notes).
\end{itemize}
 It is not working in our case. To cope the problem we need to use extra variables to construct the better classes to reduce the bias.
\end{minipage}}
\item
Assume that we got an additional information: nonresponses are distributed between the 3 groups as follows:  
\begin{itemize}
\item Group 1: 850
\item Group 2: 550
\item Group 3: 197
\end{itemize}
and they have the same voting proportion as in the response sample (the same voting proportion as in (3)).\\
Assume that there were no nonresponses and calculate weighting class adjustment estimate for the proportion of voters in this case.\\
\fcolorbox{black}{ForestGreen!20}{
\begin{minipage}[t]{0.97\linewidth}
\textbf{Solution:}
Number of sampling units in the groups will be now:
\begin{itemize}
\item Group 1: $132 + 1060 + 850 = 2042$
\item Group 2: $115 + 550 = 665$
\item Group 3: $96 + 197 = 293$
\end{itemize}
Weighting class adjustment estimate for the proportion of voters will be: $$\hat{p}_{wc} = \sum_{h=1}^3\frac{n_h}{n}\bar{y}_{hR}$$
$\hat{p}_{wc} = (2042 * 0.8893 + 665 * 0.4957 + 293 * 0.7604)/3000 = 2368/3000 = 0.789.$
\end{minipage}}
\item Under which condition weighting class adjustment estimator(4) can help largely reduce nonresponse bias? \\
\fcolorbox{black}{ForestGreen!20}{
\begin{minipage}[t]{0.97\linewidth}
\textbf{Solution:}
"The models for weighting adjustments for nonresponse are strong: In each weighting cell/poststratum, the respondents and nonrespondents are assumed to be similar, or each individual in a weighting class is assumed equally likely to respond to the survey or have a response propensity that is uncorrelated with y" (Lohr, 2019, p.342).
``The weighting class adjustment estimator approximately unbiased if within a class:
\begin{itemize}
\item the response variable $y_i$ constant
\item the response propensities $\phi_i$ the same for every unit
\item the response $y_i$ uncorrelated with the response propensity $\phi_i$"(Lecture notes).
\end{itemize}
\end{minipage}}
\end{enumerate}

\section*{Exercise 2}
\textbf{\color{ForestGreen}(R code available)} We shall estimate the mean income in a large population and take a random sample of n = 20 persons. 10 persons responded with the following income (in 1000): 600, 520, 620, 500, 380, 460, 450, 250, 400 and 780. We assume MCAR(missing completely at random) nonresponse. 
\begin{enumerate}
\item Use R to perform a hot-deck imputation for the nonresponse. Derive the standard 95\% confidence interval of the mean income in the population, based on the completed data set with observed and the imputed values.\\
\fcolorbox{black}{ForestGreen!20}{
\begin{minipage}[t]{0.97\linewidth}
\textbf{Solution:}\\
Hot-deck imputation: imputation made using the same dataset.\\
For example, we take an SRSWR of size 10 from the known responses. With a seed value in R = 2020 we impute 450, 460, 250, 600, 600, 500, 780, 460, 600, 250 so that we get values for all 20 persons\\
Average income than $$\bar{y}_{imp} = \frac{1}{n}\left(\sum_{i \in \overline{1;10}}y_i + \sum_{i \in \overline{11;20}} y_{imp;i}\right),$$ where $y_i$ - responded values, $y_{imp;i}$ - imputed values.\\
$\bar{y}_{imp} = 495.5$
$$\widehat{SE}(\bar{y}_{imp}) = (1 - f)\frac{s^2}{n}$$. \\
Since population is large $f \rightarrow 0$, then we obtain $\widehat{SE}(\bar{y}_{imp}) = 33.75$
$$CI_{imp} = \bar{y}_{imp} \pm 1.96*\widehat{SE}(\bar{y}_{imp}) = [429.35; 561.65]$$
\end{minipage}}
\item Use R to derive the standard 95\% confidence interval of the mean income in the population, based on the response sample.\\
\fcolorbox{black}{ForestGreen!20}{
\begin{minipage}[t]{0.97\linewidth}
\textbf{Solution:}\\
Making the same calculations but for only the response sample without imputation we get:\\
$\bar{y} = 496, \quad \widehat{SE}(\bar{y}) = 46.43 \quad CI_{resp} = [405.00; 587.01]$\\
Comparing thee two intervals we can see that $CI_{imp}$ is much shorter than $CI_{resp}$
\end{minipage}} 
\end{enumerate}

\section*{Exercise 3}
Investigators selected an SRS of 200 high school seniors from a population of 2 000
for a survey of television-viewing habits, with an overall response rate of $75\%$. By
checking school records, they were able to find the grade point average for the nonrespondents,
and classify the sample accordingly: 
\begin{center}
\begin{tabular}{lcccc}
& Sample & Number of & \multicolumn{2}{c}{Hours of TV}\\
GPA& size & respondents & $\bar{y}_{cR}$ & $s_{cR}$ \\
\hline
3.00–4.00 & 75& 66& 32& 15\\
2.00–2.99& 72& 58& 41& 19\\
Below 2.00& 53& 26& 54& 25\\
\hline
Total & 200 & 150 && \\
\end{tabular}
\end{center}
\begin{enumerate}
\item What is the estimate for the average number of hours of TV watched per week if only respondents are analyzed? What is the standard error of the estimate?\\
\fcolorbox{black}{ForestGreen!20}{
\begin{minipage}[t]{0.97\linewidth}
\textbf{Solution:}
Find $\bar{y}_R$ using only the respondent set and provide a standard error of your estimate.
$$\bar{y}_R=\frac{1}{n_R}\sum_{c=1}^3 n_{cR}\bar{y}_{cR}=\frac{1}{150}[66(32)+58(41)+26(54)]=39.3\cdot$$
$$\widehat{SE}(\bar{y}_R)=\sqrt{\Big( 1-\frac{150}{2\,000}\Big)\frac{s_R^2}{150}},\quad s_R^2=\frac{1}{n_R-1}\Big(\sum_{c=1}^3\sum_{i=1}^{n_{cR}}y_i^2-n_R \bar{y}_R^2\Big),$$
{\small
\begin{eqnarray*}
\sum_{c=1}^3\sum_{i=1}^{n_{cR}}y_i^2&=&\sum_{c=1}^3(n_{cR}-1)s_{cR}^2+\sum_{c=1}^3n_{cR}\bar{y}_{cR}^2\\
&=&[65(15^2)+57(19^2)+25(25^2)]+[66(32^2)+58(41^2)+26(54^2)] = 291\,725\cdot
\end{eqnarray*}
}
\vspace{-1cm}
\begin{itemize}
\item $s_R^2=\frac{1}{149}[291\,725 - 150(39.3^2)]=403.6$
\item $\widehat{SE}(\bar{y}_R)=1.58$
\end{itemize}

\end{minipage}}
\item Perform a $\chi^2$ test for the null hypothesis that the three GPA groups have the same
response rates. What do you conclude? What do your results say about the type of missing data: Do you think the data are MCAR? MAR? Nonignorable?\\
\fcolorbox{black}{ForestGreen!20}{
\begin{minipage}[t]{0.97\linewidth}
\textbf{Solution:}
Perform a $\chi^2$ test for the null hypothesis that the three GPA groups have the same
response rates, that is, $H_0:\phi_c=0.75$. 
{\small
\begin{center}
\begin{tabular}{lccc}
GPA & Respondents & Non respondents & Total\\
\hline
3.00-4.00 &66& 9 &75\\
2.00-2.99 &58& 14 &72\\
Below 2.00& 26& 27& 53\\
\hline
Total& 150 &50& 200
\end{tabular}
\end{center}
}
\vspace{-0.5cm}
\begin{eqnarray*}
\chi^2 &=&\sum_{c}\frac{(\text{observed}_c-\text{expected}_c)^2}{\text{expected}_c}\\
&=&\frac{[66-0.75(75)]^2}{0.75(75)}+\frac{[9-0.25(75)]^2}{0.25(75)}+\cdots + \frac{[27-0.25(53)]^2}{0.25(53)}\\
&=&1.69 + 5.07 + 0.30 + 0.89 + 4.76 + 14.27 = 26.97\cdot
\end{eqnarray*}
The $\chi^2$ text statistics is $26.97$. The $p$-value for the area to the right of $\chi^2=26.97$ under a $\chi^2$ distribution with degree-of-freedom $\text{df}=2$ is $1.4\times 10^{-6}<<\alpha=0.05$. Thus, there is a strong evidence against the null hypothesis that the three groups have the same response rates. 
The hypothesis test indicates that the nonresponse is not MCAR, because response
rates appear to be related to GPA. We do not know whether the nonresponse is
MAR, or whether is it nonignorable.
\end{minipage}}
\item Perform a one-way ANOVA analysis to test the null hypothesis that the three GPA groups have the same mean level of television viewing. What do you conclude? Does your ANOVA analysis indicate that GPA would be a good variable for constructing weighting cells? Why, or why not?\\
\fcolorbox{black}{ForestGreen!20}{
\begin{minipage}[t]{0.97\linewidth}
\textbf{Solution:}
The ANOVA table is given as follows:
{\small
\begin{center}
\begin{tabular}{llll}
Source & df & Sum of squares, estimated & Mean square, est.  \\
\hline
Between groups& $C-1$ &$\widehat{SSB}=\sum_{c=1}^3\sum_{i=1}^{n_{cR}}(\bar{y}_{cR}-\bar{y}_R)^2$& $\widehat{MSB}$\\
Within groups& $n_R-C$ & $\widehat{SSW}=\sum_{c=1}^3\sum_{i=1}^{n_{cR}}(y_i-\bar{y}_{cR})^2$& $\widehat{MSW}$\\
\hline
Total, about mean & $n_R-1$& $\widehat{SST}=\sum_{c=1}^3\sum_{i=1}^{n_{cR}}(y_i-\bar{y}_R)^2$& $\widehat{MST}=s_R^2$\\
\end{tabular}
\end{center}
}
$$\widehat{SSB}=\sum_{c=1}^3\sum_{i=1}^{n_{cR}}(\bar{y}_{cR}-\bar{y}_R)^2=\sum_{c=1}^3{n_{cR}}(\bar{y}_{cR}-\bar{y}_R)^2=9\,303.1$$
$$\widehat{MST}=\frac{1}{n_R-1}\sum_{c=1}^3\sum_{i=1}^{n_{cR}}(y_i-\bar{y}_R)^2=s_R^2=403.6$$
$$\widehat{SSW}=149s_R^2-9\,303.1=50\,833.3,\quad \widehat{MSW}=\frac{50\,833.3}{147}=345.8$$
\begin{center}
\begin{tabular}{lrrrrr}
Source & df & $\hat{SS}$  & $\hat{MS}$ & F & $p$-value\\
\hline
Between groups& 2 &9 303.1& 4\,651.5& 13.4& $4.5\times 10^{-6}$\\
Within groups& 147& 50 833.3 &345.8&&\\
\hline
Total, about mean & 149& 60 136.4&&&\\
\end{tabular}
\end{center}
There is a strong evidence against the null hypothesis that the three GPA groups have the same mean level of TV viewing. Based on the F-test here and the $\chi^2$ test in (2), both the nonresponse rate and the TV viewing seems to be related to GPA. Therefore, it would be reasonable to use GPA groups for weighting class adjustment, or poststratification. 
\end{minipage}}
\item Use the GPA classification to adjust the weights of the respondents in the sample. What is the weighting class estimate of the average viewing time?\\
\fcolorbox{black}{ForestGreen!20}{
\begin{minipage}[t]{0.97\linewidth}
\textbf{Solution:}
Use the GPA classification for weighting class adjustment, and find $\bar{y}_{wc}$.
$$\bar{y}_{wc}=\frac{\sum_{c=1}^3\sum_{i=1}^{n_{cR}} d_i\hat{\phi}_c^{-1}y_i}{\sum_{c=1}^3\sum_{i=1}^{n_{cR}} d_i\hat{\phi}_c^{-1}}=\frac{\sum_{c=1}^3n_{cR}\hat{\phi}_c^{-1}\bar{y}_{cR}}{\sum_{c=1}^3n_{cR}\hat{\phi}_c^{-1}},$$
where $d_i=2\,000/200=10$ and $ \hat{\phi}_c=n_{cR}/n_c$.
$$\bar{y}_{wc}=\frac{66(75/66)32+58(72/58)41+26(53/26)54}{66(75/66)+58(72/58)+26(53/26)}=\frac{8\,214}{200}=41.07\cdot$$\\

Here, the weights after the weighting class adjustment are given by $w_{i;wc}=d_i\hat{\phi}^{-1}=10n_c/n_{cR}, \, i\in S_{cR}$. We have $w_{i;wc}=11.36$, $w_{i;wc}=12.41$, and $w_{i;wc}=20.38$, for the units in classes $c=1$, $c=2$, and $c=3$, respectively. \\

The estimated average time of TV viewing is higher after the weighting class adjustment, that is, $\bar{y}_{wc}=41.07>\bar{y}_R=39.3$. This is due to that $\bar{y}_{cR}$ is the highest for the class where the nonresponse rate is the highest.   
\end{minipage}}
\item The population counts are 700 students with GPA between 3 and 4; 800 students with GPA between 2 and 3; and 500 students with GPA less than 2. Use these population counts to construct a poststratified estimate of the mean viewing time.\\
\fcolorbox{black}{ForestGreen!20}{
\begin{minipage}[t]{0.97\linewidth}
\textbf{Solution:}
Let $N_1=700$, $N_2=800$, and $N_3=500$. Use a poststratified estimator to estimate $\bar{Y}$.
$$\bar{y}_{post}=\frac{\sum_{h=1}^3\sum_{i=1}^{n_{hR}} d_i\hat{\phi}_h^{-1}y_i}{\sum_{h=1}^3\sum_{i=1}^{n_{hR}} d_i\hat{\phi}_h^{-1}}=\frac{\sum_{h=1}^3N_h\bar{y}_{hR}}{\sum_{h=1}^3N_h},$$
where $d_i=2\,000/200=10$, and
$$\hat{\phi}_h=\frac{\hat{N}_{hR}}{N_h}=\frac{\sum_{i=1}^{S_{hR}} d_i}{N_h}=\frac{n_{hR}(N/n)}{N_h}\cdot$$.
$$\bar{y}_{post}=\frac{700(32)+800(41)+500(54)}{700+800+500}=\frac{82\,200}{2\,000}=41.1\cdot$$
Here, the poststratified weights are given by 
$$w_{i;post}=d_i\hat{\phi}_h^{-1}=d_i\frac{N_h}{n_{hR} (N/n)}=\frac{N_h}{n_{hR}}, \, i\in S_{hR},$$

We have $w_{i;post}=10.61$, $w_{i;post}=13.79$, and $w_{i;post}=19.23$, for the units in poststrata $h=1$, $h=2$, and $h=3$, respectively. \\

Note that $\sum_{i \in S_{hR}}w_{i;post}=N_h$, that is, $66(10.61)=700$, $58(13.79)=800$, and  $26(19.23)=500$.   

\end{minipage}}
\end{enumerate}

\section*{Exercise 4}
\textbf{\color{ForestGreen}(R code available)} The American Statistical Association (ASA) studied whether it should offer a certification designation for its members, so that statisticians meeting the qualifications
could be designated as “Certified Statisticians.” In 1994, the ASA surveyed its membership about this issue, with data in file certify.dat. The survey was sent to all 18 609
members; 5 001 responses were obtained. Results from the survey were reported in the October 1994 issue of Amstat News\\
Assume that in 1994, the ASA membership had the following characteristics: $55\%$
have PhD’s and $38\%$ have Master’s degrees; $29\%$ work in industry, $34\%$ work in
academia, and $11\%$ work in government. The cross-classification between education
and workplace was unavailable.
\begin{enumerate}
\item What are the response rates for the various subclasses of ASA membership? Are the nonrespondents MCAR? Do you think they are MAR?
ASA survey of certification designation for its members. $n=N=18\,609$ and $n_R=5\,001$. \\
\fcolorbox{black}{ForestGreen!20}{
\begin{minipage}[t]{0.97\linewidth}
\textbf{Solution:}
\begin{center}
\begin{tabular}{lr}
& Percentage of ASA members \\
& in 1994 ($p_c$)\\ 
\hline
PhD & $0.55$ \\
Master's &$0.38$\\
Other/unknown & $0.07$\\
\hline
Industry &$0.29$\\ 
Academia &$0.34$\\
Government &$0.11$\\
Other/unknown &$0.26$\\
\end{tabular}
\end{center}
Find response rates for the subclasses of ASA membership.
{\small
\begin{center}
\begin{tabular}{lrrr}
& $n_c$ &$n_{cR}$ & $\phi_c\,(\%)$ \\
\hline
PhD & 10 235& 3 036& 30\\
Master's &7 071& 1 640 &23\\
Other/Unknown& 1 303& 325& 25\\
\hline
Industry& 5 397 &1 809 &34\\
Academia &6 327& 2 221 &35\\
Government& 2 047 &880& 43\\
Other/Unknown &4 838& 91& 2\\
\end{tabular}
\end{center}
}
Here, $n_c=18\,609p_c$, $\phi_c=n_{cR}/n_c$, and $n_{cR}$ are obtained from the survey data.\\
The response rates are quite low. The nonresponse does not appear to be MCAR, as it differs by degree and by type of employment. It may not be sure that it is MAR either. \\
\end{minipage}}
\item Use raking to adjust the weights for the six cells defined by education (PhD or non-PhD) and workplace (industry, academia, or other). Start with an initial weight of $18\,609/5\,001$ for each respondent. What assumptions must you make to use raking?\\
\\
\fcolorbox{black}{ForestGreen!20}{
\begin{minipage}[t]{0.97\linewidth}
\textbf{Solution:}
The number of respondents by cells is given as follows:
\begin{center}
\begin{tabular}{lrrrrr}
&Industry & Academia& Other&\vline & \\
\hline
PhD& 798& 1 787& 451& \vline &3 036\\
non-PhD& 1 011& 434 &520& \vline &1 965\\
\hline
&1 809 &2 221& 971&\vline & 5 001 \\
\end{tabular}
\end{center}

Using $w_i=18\,609/5\,001=3.72$ as initial weights, we obtain initial estimates of population totals by cells as follows:
\begin{center}
\begin{tabular}{lrrrrr}
&Industry & Academia& Other&\vline & \\
\hline
PhD &2 969.4 &6 649.5 &1 678.2& \vline &11 297.1\\
non-PhD &3 762.0 &1 614.9 &1 934.9&\vline & 7 311.9\\
\hline
&6 731.4 &8 264.5& 3 613.1& \vline &18 609.0 \\
\end{tabular}
\end{center}
Adjusting rows by the population counts $10\,235$ and $8\,374$ for PhD and non-PhD, respectively, we obtain
\begin{center}
\begin{tabular}{lrrrrr}
&Industry &Academia& Other&\vline &\\
\hline
PhD &2 690.2& 6 024.4 &1 520.4& \vline &10 235\\
non-PhD &4 308.5 &1 849.5 &2216.0 &\vline &8 374\\
\hline
&6 998.7& 7 873.9 &3 736.4 &\vline &18 609\\
\end{tabular}
\end{center}
Now adjusting columns by the population counts $5\,397	$, $6\,327$, and $6\,885$ for industry, academia and other, respectively, we obtain
\begin{center}
\begin{tabular}{lrrrrr}
&Industry& Academia &Other&\vline& \\
\hline
PhD &2 074.6& 4 840.8 &2 801.6 &\vline& 9 717.0\\
non-PhD &3 322.4 &1 486.2 &4 083.4 &\vline& 8 892.0\\
\hline
&5 397.0 &6 327.0 &6 885.0&\vline& 18 609.0\\
\end{tabular}
\end{center}
After iterative raking, the final cell estimates are given by
\begin{center}
\begin{tabular}{lrrrrr}
&Industry &Academia& Other&\vline \\
\hline
PhD& 2 239.2& 4 980.6 &3 015.2 &\vline &10 235.0\\
non-PhD &3 157.8 &1 346.4 &3 869.8&\vline & 8 374.0\\
\hline
&5 397.0& 6 327.0& 6 885.0 &\vline &18 609.0\\
\end{tabular}
\end{center}
Final weights by cells are given as follows:
\begin{center}
\begin{tabular}{lrrr}
&Industry &Academia& Other \\
\hline
PhD& 2.81&	2.79&	6.69\\
non-PhD&3.12	&3.10&	7.44\\
\end{tabular}
\end{center}
\begin{itemize}
\item The weights in the ``Other" employment category are dramatically increased by raking.
\item The raking weights are used to estimate population quantities. 
\item For raking, we assume that the response probabilities depend only on the marginal counts, not on the cell counts. Given marginal counts, the response $y_i$ is uncorrelated with the response propensity $\phi_i$.
\end{itemize}
\end{minipage}}
\item Can you conclude from this survey that a majority of the ASA membership opposed certification in 1994? Why, or why not?\\
\fcolorbox{black}{ForestGreen!20}{
\begin{minipage}[t]{0.97\linewidth}
\textbf{Solution:}
{\small
\begin{center}
\begin{tabular}{lrr}
&Without& With raking\\
&weights& weights\\
\hline
No response &0.2& 0.3\\
Yes& 26.4 &25.8\\
Possibly& 22.3 &22.3\\
No opinion& 5.4 &5.4\\
Unlikely& 6.7& 6.9\\
No &39.0 &39.3\\
\end{tabular}
\end{center}
}
\begin{itemize}
\item We can not rely on the conclusion that the majority of the ASA members are against certification as the nonresponse rates are very high. 
\end{itemize}
\end{minipage}}
\end{enumerate} 


\end{document}