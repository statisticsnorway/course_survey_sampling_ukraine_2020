\documentclass[12pt]{article}

% Packages
\usepackage[margin=0.6in]{geometry}
\usepackage{graphicx}
\usepackage{amsmath}
\usepackage{float}
\usepackage[dvipsnames]{xcolor}
\usepackage{etoolbox}
\usepackage{soul}
%\usepackage{tcolorbox}
%\definecolor{block-gray}{gray}{0.85}

\pagestyle{plain}


\begin{document}

\title{Exercise ark \#2.\\ Stratified sampling}
\author{O\u{g}uz--Alper, Melike \& Pekarskaya, Tatsiana, Statistics Norway}
\maketitle

\section*{Exercise 0}

\begin{enumerate}
\item Explain in your own words what is stratified sample?
\item What can be reasons to use a stratified sample rather than SRS?
\end{enumerate}

\section*{Exercise 1}

Consider a population of 6 students. Suppose we know the test scores of the students
to be
\begin{center}
\begin{tabular}{lrrrrrrr}
Student & \vline & 1& 2& 3& 4& 5& 6\\
\hline
Score & \vline &  66& 59& 70& 83& 82& 71\\
\end{tabular}
\end{center}
\begin{enumerate}
\item Find the mean $\bar{y}_U$ and variance $S^2$ of the population.
\item How many SRS’s of size $4$ are possible?
\item List the possible SRS’s. For each, find the sample mean. Find $V(\bar{y})$, using $$V(\bar{y})=(1-n/N)n^{-1}S^2\cdot$$
\item Now let stratum one consist of students 1–3, and stratum two consist of students 4–6. How many stratified random samples of size $4$ are possible in which $2$ students are selected from each stratum?
\item List the possible stratified random samples. Which of the samples from (3) cannot occur with the stratified design defined in (4)?
\item Find $\bar{y}_{str}$ for each possible stratified random sample. Find $V(\bar{y}_{str})$, and compare it to $V(\bar{y})$. \hfill (Lohr, 2019, p.102)
\end{enumerate}

\section*{Exercise 2}
A stratified sample is being designed to estimate the prevalence p of a rare characteristic,
say the proportion of residents in Milwaukee, Wisconsin, who have Lyme
disease. Stratum one, with $N_1$ units, has a high prevalence of the characteristic; stratum
two, with $N_2$ units, has low prevalence. Assume that the cost to sample a unit (for
example, the cost to select a person for the sample and determine whether he or she
has Lyme disease) is the same for each stratum, and that at most $2\,000$ units are to be
sampled.
\begin{enumerate}
\item  Let $p_1$ and $p_2$ be the proportions in stratum one and stratum two with the rare characteristic.
If $p_1=0.10$, $p_2=0.03$, and $N_1/N = 0.4$, what are $n_1$ and $n_2$ under
optimal allocation?
\item If $p_1=0.10$, $p_2=0.03$, and $N_1/N = 0.4$, what is $V(\hat{p}_{str})$ under proportional
allocation? Under optimal allocation? What is the variance if you take an SRS of
$2\,000$ units from the population? \hfill (Lohr, 2019, p.110)
\end{enumerate}


\section*{Exercise 3}
\textbf{\color{ForestGreen}(R code available)} Consider the same problem as in exercise 4 previous section. In the SRS of 50 faculty members not all the departments were represented. The SRS contained several faculty members from psychology and from chemistry but none from the foreign languages. It was therefore decided to take a stratified simple random sample, using the areas biological sciences, physical sciences, social sciences and humanities as the strata. Proportional allocation was used in this sample. The distribution of the strata for population and sample are given below
	
\begin{center}
\begin{tabular}{lcc}
Stratum & Number of faculty & Number of faculty members\\
& members in the stratum & in the samples\\
\hline
1 - Biological sciences & 102 & 7\\
2 - Physical sciences & 310 & 19\\
3 - Social sciences & 217 & 13\\
4 - Humanities & 178 & 11\\
\hline
Total & 807 & 50\\
\end{tabular}
\end{center}

The data from the stratified sample turned out to be
\begin{center}
\begin{tabular}{c|cccc}
Number of & Number of refereed publications \\
faculty members & Biological & Physical & Social & Humanities\\
\hline
0&	1	&10	&9	&8\\
1&	2	&2	&0	&2\\
2&	0	&0	&1	&0\\
3&	1	&1	&0	&1\\
4&	0	&2	&2	&0\\
5&	2	&1	&0	&0\\
6&	0	&1	&1	&0\\
7&	1	&0	&0	&0\\
8&	0	&2	&0	&0\\

\end{tabular}
\end{center}

\begin{enumerate}
\item Estimate the total number of refereed publications by faculty members in the college and compute the standard error.
\item How does the result from part (3) compare with the result from the SRS in Exercise 4 previous session?
\item Estimate the proportion of faculty with no refereed publications and compute the standard error and 95\% confidence interval.
\item Compare the result in part (3) with the result from the SRS in Exercise 4 previous session.
\item Did the stratification increase precision for the two estimation problems considered in parts (1) and (3) ? Explain why you think it did or did not.\hfill{(Lohr 2019 pp 103-104)}
\end{enumerate}


\section*{Exercise 4}
A public opinion researcher has a budget of \$20 000 for taking a survey. She knows that 90\% of all households have known telephone numbers. Telephone interviews cost \$10 per household; in-person interviews cost \$30 each if all interviews are conducted in person and \$40 each if only households with unknown telephone numbers are interviewed in person (because there will be extra travel costs). Assume that the variances in phone and nonphone strata are similar and that the fixed costs are $c_0 = \$5\,000$.   
\begin{enumerate}
\item How many households should be interviewed in each group if all households are interviewed in person ?
\item How many households should be interviewed in each stratum if households with a known telephone number are contacted by telephone and the other households are contacted in person?
\item Which of the two data collection methods would you choose? Give a justification for your answer.\hfill{Lohr, 2019, p.104}
\end{enumerate}


\section*{Exercise 5}
Burnard (1992) sent a questionnaire to a stratified sample of nursing tutors and students in Wales, to study what the tutors and students understood by the term \emph{experiential learning}. The population size and sample size obtained for each of the four strata are
given below:
\begin{center}
\begin{tabular}{lrr}
Stratum & Population size & Sample size \\
\hline
General nursing tutors (GT) &150 &109\\
Psychiatric nursing tutors (PT)&34 &26\\
General nursing students (GS) &2 680& 222\\
Psychiatric nursing students (PS)&570& 40\\
\hline
Total & 3\,434 &397 \\
\end{tabular}
\end{center}
Respondents were asked which of the following techniques could be identified as experiential learning methods; the number of students in each group who identified
the method as an experiential learning method are given below:
\begin{center}
\begin{tabular}{lrrrrr}
Method & \vline & GS & PS & PT & GT \\
\hline
Role play &\vline & 213 &38& 26 &104\\
Problem solving activities&\vline & 182& 33& 22& 95\\
Simulations &\vline &95 &20 &22 &64\\
Empathy-building exercises &\vline &89 &25& 20& 54\\
Gestalt exercises &\vline &24& 4& 5& 12\\
\end{tabular}
\end{center}
Estimate the overall percentage of nursing students and tutors who identify ``Role play" techniques as “experiential learning”. Be sure to give standard errors for your estimate. \hfill (Lohr, 2019, p.106)

\end{document}